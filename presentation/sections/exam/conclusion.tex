\section{Заключение}

\begin{frame}{Сравнительный анализ методов}
    \textbf{Ключевые отличия:}
    \begin{itemize}
        \item \textbf{PCA:} базовый линейный метод, идеален для анализа линейных зависимостей.
        \item \textbf{KPCA:} позволяет работать с нелинейной структурой данных, но требует осторожной настройки.
        \item \textbf{AE:} предоставляет большую гибкость благодаря нейронным сетям.
        \item \textbf{VAE:} расширяет автоэнкодеры за счет вероятностной модели, идеально подходит для задач генерации данных и анализа скрытых переменных.
    \end{itemize}
\end{frame}

\begin{frame}{Заключение}
    \textbf{Современные вызовы:} Работа с высокоразмерными данными требует гибких и мощных инструментов. Методы понижения размерности предоставляют исследователям возможность:
    \begin{itemize}
        \item \textbf{Повышения информативности} данных.
        \item \textbf{Улучшения эффективности} алгоритмов машинного обучения.
        \item \textbf{Удобства визуализации} данных.
    \end{itemize}

    \textbf{Вывод:} Методы понижения размерности предоставляют исследователям мощный арсенал для анализа данных, повышая информативность, эффективность и удобство визуализации.
\end{frame}
