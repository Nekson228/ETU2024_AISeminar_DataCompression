\section{Заключение}

\begin{frame}[allowframebreaks]{Сравнительный анализ методов: особенности применения}
    \textbf{Ключевые отличия:}
    \begin{itemize}
        \item \textbf{PCA:} базовый линейный метод, идеален для анализа небольших и линейных зависимостей. Часто используется для визуализации и как отправная точка в анализе данных.
        \item \textbf{KPCA:} позволяет работать с нелинейной структурой данных, но требует осторожного выбора ядерной функции и гиперпараметров. Идеален для задач распознавания образов и биоинформатики.
        \item \textbf{AE:} предоставляет большую гибкость благодаря нейронным сетям. Находит применение в обработке данных и сложных задачах анализа.
        \item \textbf{VAE:} расширяет автоэнкодеры за счет вероятностной модели, идеально подходит для задач генерации данных и анализа латентных переменных.
    \end{itemize}
    \textbf{Дополнительно:}
    \begin{itemize}
        \item Все методы обладают уникальными преимуществами и ограничениями, что делает их подходящими для разных классов задач.
        \item Выбор метода зависит от структуры данных, целей анализа и доступных вычислительных ресурсов.
    \end{itemize}
\end{frame}

\begin{frame}[allowframebreaks]{Заключение}
    \textbf{Современные вызовы:} Работа с высокоразмерными данными требует гибких и мощных инструментов.

    \textbf{Основные итоги:}
    \begin{itemize}
        \item \textbf{Линейные методы} (например, PCA) остаются незаменимыми благодаря своей простоте и эффективности.
        \item \textbf{Нелинейные подходы}, такие как KPCA, открывают возможности работы с более сложными структурами данных.
        \item \textbf{Автоэнкодеры} обеспечивают исключительную гибкость для задач генерации данных и анализа скрытых зависимостей.
    \end{itemize}

    \framebreak

    \textbf{Рекомендации по выбору:}
    \begin{itemize}
        \item Для интерпретируемости и быстродействия – линейные методы.
        \item Для работы с нелинейными структурами – Kernel PCA.
        \item Для генерации данных – автоэнкодеры и их вариации.
    \end{itemize}

    \textbf{Вывод:} Методы понижения размерности предоставляют исследователям мощный арсенал для анализа данных, повышая информативность, эффективность и удобство визуализации.
\end{frame}
